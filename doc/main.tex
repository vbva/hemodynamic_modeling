\documentclass[12pt]{article}

\usepackage[russian]{babel}
\usepackage[utf8]{inputenc}
\usepackage[T2A]{fontenc}
\usepackage{amsmath}
\usepackage{indentfirst}
\usepackage[colorlinks]{hyperref}


\title{Одномерная модель кровотока в сети сосудов: описание программы}
\author{Р.Янбарисов}
\date{}

\begin{document}
\maketitle

\section{Математическая модель}\label{sec:model}

\subsection{Управляющие соотношения}\label{sec:gov}

Закон сохранения массы и импульса в переменных $A$ (площадь поперечного сечения сосуда), $U$ (скорость кровотока), $P$ (давление в сосуде):
\begin{align}\label{eq:gov}
	\frac{\partial A}{\partial t} + \frac{\partial AU}{\partial x} &= 0 \\
	\frac{\partial U}{\partial t} + U \frac{\partial U}{\partial x} + \frac{1}{\rho}\frac{\partial P}{\partial x} &= \frac{f}{\rho A} \\
\end{align}
Замыкающее соотношение, связывающее давление с площадью сосуда:
\begin{align}\label{eq:pressure}
	P = P_{dia} + \frac{\beta}{A_{dia}} \left( \sqrt{A} - \sqrt{A_{dia}} \right)
\end{align}
Здесь $P_{dia}$ и $A_{dia}$ -- известные из входных данных диастолические давление и площадь, $\beta = \frac{4}{3} \sqrt{\pi} E h$ -- множитель, учитывающий свойства несжимаемой стенки сосуда толщиной $h$ из линейного упругого материала с модулем Юнга $E$.

Сосуд считается осесимметричной трубкой с переменным вдоль его длины радиусом круга его поперечного сечения.
Предполагается, что скорость также осесимметрична и определяется в зависимости от радиальной координаты $\xi$ как
\begin{align*}
	u(x,\xi,t) = U(x,t) \frac{\zeta+2}{\zeta} \left[ 1 - \left(\frac{\xi}{r}\right)^\zeta \right].
\end{align*}
Здесь $r(x,t)$ -- радиус просвета сосуда, $\zeta$ -- параметр, определяющий профиль \footnote{при $\zeta = 2$ получается профиль Пуазейля}.

Для заданной таким образом скорости сила трения $f$ в \eqref{eq:gov} определяется следующим образом:
\begin{align}\label{eq:friction}
	f = -2 (\zeta+2) \mu \pi U.
\end{align}
 
\subsection{Граничные и начальные условия}\label{sec:conditions}

Система уравнений \eqref{eq:gov}, \eqref{eq:pressure} является гиперболической.

В качестве начальных условий задаются постоянные поля скоростей и площади: $A|_{t=0} = A_0 = const$, $U|_{t=0} = U_0 = const$. Давление определяется из соотношения \eqref{eq:pressure}.

Реализованы следующие варианты граничных условий.

\begin{itemize}
	\item Условие втока
		\begin{align}\label{eq:inlet}
			Q(x_{in},t) = Q_{in}(t).
		\end{align}
		Возможно задание функции $Q_{in}(t)$ как в виде аналитической функции, так и таблицей значений (в этом случае используется линейная интерполяция \footnote{более приемлемым вариантом может являться интерполяция сплайнами для большей гладкости}).
	\item Условие вытока
		\begin{itemize}
			\item Заданное постоянное давление на вытоке
				\begin{align}\label{eq:outflow_pressure}
					P(x_{out}, t) = P_{out} = const
				\end{align}
			\item Условие типа Виндкесселя
				\begin{align}\label{eq:windkessel}
					Q\left(1+\frac{R_1}{R_2}\right) + C R_1 \frac{\partial Q}{\partial t} = \frac{P - P_{out}}{R_2} + C \frac{\partial P}{\partial t}.
				\end{align}
		\end{itemize}
	\item Условие на стыке $n$ сосудов -- сохранение массы и полного интеграла Бернулли
		\begin{align}\label{eq:junction}
			\sum\limits_{i=1}^{n} Q_i &= 0; \\
			P_1 + \frac{\rho U_1^2}{2} &= P_j + \frac{\rho U_j^2}{2}, j = 2, \dots, n.
		\end{align}
\end{itemize}

Во всех случаях выше поток определяется через неизвестные задачи $Q = AU$.

Условие типа Виндкесселя определяется параметрами $P_{out}$ (давление на выходе из сосуда), $R_1$ и $R_2$, $C$. Последние три параметра определяются характеристиками сосудов за пределами рассматриваемой системы сосудов, не разрешенных одномерной моделью, такими, как вместимость ($C$), и другими геометрическими характеристиками ($R_1$, $R_2$).

\section{Численный метод}\label{sec:numerical}

Подробное описание доступно в диссертации (\href{https://www.inm.ras.ru/wp-content/uploads/dis-sovet/disser/\%D0\%93\%D0\%B0\%D0\%BC\%D0\%B8\%D0\%BB\%D0\%BE\%D0\%B2_\%D0\%94\%D0\%B8\%D1\%81\%D1\%81\%D0\%B5\%D1\%80\%D1\%82\%D0\%B0\%D1\%86\%D0\%B8\%D1\%8F.pdf}{ссылка})
 Гамилова Тимура Мударисовича, к.ф.-м.н., глава 3.

Дискретизация граничных условий основана на представлении неизвестной скорости на границе через неизвестную площади $U = \alpha A + \beta$, где коэф-ты $\alpha$ и $\beta$ определяются из условий совместности системы гиперболических уравнений. 

В этом случае неизвестная площадь $A$ сосуда на его границе определяется в зависимости от типа граничного условия.

\begin{itemize}
	\item Условие втока сводится к квадратному уравнению относительно $A$.
	\item Условие заданного постоянного давления на вытоке сводится к вычислению обратной функции $A = A(P)$ относительно управляющего соотношения \eqref{eq:pressure}.
	\item Условие на стыке $n$ сосудов сводится к системе $n$ нелинейных уравнений относительно площадей $A_i$ каждого из сосудов на стыке. Система решается методом Ньютона с решением на предыдущем временном слое в качестве начального приближения.
	\item Условие типа Виндкесселя сводится к нелинейному уравнению относительно неизвестной $A$, которое так же, как и для условия на стыке, решается методом Ньютона.
\end{itemize}

\section{Структура кода}\label{sec:program}

Программа организована с использованием ООП и классов для различных сущностей.

Класс обработки параметров отвечает за считывание параметров из файла в формате json и удобство их использования.

Класс сосуда отвечает за решение дискретизированных уравнений \eqref{eq:gov} во внутренних точках его расчетной сетки. Сама расчетная сетка в виде массива определена внутри объектов этого класса.

Класс граничного условия отвечает за решение дискретизированных уравнений для граничных условий из раздела \ref{sec:conditions}, \ref{sec:numerical} и обновление значений в массивах сосуда/сосудов, для которых это условие определено.

Управляющий класс модели объединяет сосуды и граничные условия и отвечает за постпроцессинг решения (если необходимо сохранить решение или посчитать его характеристики).

Поскольку граничное условие выставляется на обоих концах каждого сосуда, вводится дополнительный параметр \textrm{BC\_SIDE}, определяющий, какое из значений массива расчетной сетки (первое или последнее) должно обновляться данным граничным условием.

Предполагается, что граничное условие втока обновляет первое значение расчетной сетки соответствующего сосуда. 
Для однозначного определения этого параметра на всех границах сосудов, включая стыки, вводится еще один параметр ориентации \textrm{ori} на каждом сосуде. Этот параметр равен 1, если направление увеличения индекса массива расчетной сетки совпадает с осью Ox вдоль сосуда, и равен -1 в противном случае.

\end{document}